\subsection{Photography Oral: (Day2)}
\subsubsection{Photon-Flooded Single-Photon 3D Cameras}
    1. {\bf Problem:} Sunlight disturb Ambient light \\
    2. {\bf Solution:}  \\
        1. Find Optimal Filtering: low distortion, high SNR \\
    3. {\bf Result:} \\
        1. Long Range Low Power 3D Imaging \\
\subsubsection{High Flux Passive Imaging with Single-Photon Sensors}
    1. SPADs \\
    2. {\bf Problem:} Noise PF-SPAD, High Flux fail to catch Photons \\
    3. {\bf Solution:} PF-SPAD Sensor can catch High Dynamic Range \\
    4. {\bf Result:} PF-SPAD: SPADs as General-Purpose, Passive Sensors. \\
\subsubsection{Acoustic Non Line of sight Imaging}
    1. {\bf Problem:} Expensive \\
    2. {\bf Solution:} Acoustic, cheap. Use Wall as mirror. \\
\subsubsection{Steady-State Non Line of Sight Imaging}
    1. {\bf Problem:} Large setup size. Expensive \\
    2. {\bf Solution:} \\
\subsubsection{A Theory of Fermat Paths for Non line of sight shape reconstruction (CVPR19 best paper)}
    1. {\bf Problem:} Non line of sight \\
    2. {\bf Solution:} Scanning the wall \\
        1. Fermat path lengths = discontinuities \\
        2. Fermat path = specula or boundary \\
        3. Add regularization term \\
        4. Optical Coherence Tomography to high resolution \\
    3. {\bf Result:}  \\
        1. High resolution \\
\subsubsection{Projector Photometric Compensation}
    1. {\bf Problem:} Projection distort image \\
    2. {\bf Contribution:} \\
        1. CompenNet CNN \\
        2. Premarin method \\
        3. Benchmark \\
    3. {\bf Solution:} \\
        1. Capture Surface image and camera image \\
        2. Train CompenNet \\
    4. {\bf Result:} \\
        1. Surpass State of the art \\
\subsubsection{Bringing a Blurry Frame Alive at High Frame-Rate With an Event Camera}
    1. {\bf Problem:} Event camera likely to capture blur image \\
    2. {\bf Solution:}  \\
        1. Double Integral while feeding in event. \\
        2. Find gradient descent by Fibonacci search \\
    3. {\bf Result:} \\
        1. Sharp Video Sequence \\
\subsubsection{Bringing Alive Blurred Moments}
    1. {\bf Problem:} unsuitable for realtime,Too many unknowns, estimate only in one image \\
    2. {\bf Solution:} \\
        1. Learn Extract motion from a sharp frame sequence \\
        2. Recurrent Video Encoder decoder \\
\subsubsection{Learning to Synthesize Motion Blur}
    1. {\bf Problem:} Motion During Exposure Causes Blur, accidental , Purposefully  \\
        1. Optical Flow Blur unwanted object  \\
    2. {\bf Solution:} Synthetic Motion Blur \\
        1. Train Model to synthesize \\
        2. Could learn occlusion \\
        3. Training Date Generation: \\
            1. Train Frame interpolation network \\
            2. Average for motion blur \\
    3. {\bf Result:} \\
        1. High dB \\
        2. Short Time \\
        3. Handling Complex Motions Better than Optical flow method \\
\subsubsection{Underexposed Photo Enhancement Using Deep Illumination Estimation}
    1. {\bf Problem:} Pixel-wise mapping has limitation \\
    2. {\bf Solution:}  \\
        1. Illumination Map \\
        2. Smoothness loss \\
\subsubsection{Blind Visual Motif Removal From a Single Image}
    1. {\bf Solution:}  \\
        1. 1 encoder, 3 decoders \\
        2. A bunch of losses \\
    2. {\bf Test Result:} \\
\subsubsection{Non-Local Meets Global: an integrated Paradigm for Hyper-spectral Denoising}
    1. {\bf Problem:}  \\
        1. More spectral bands, more computation burden \\
    2. {\bf Solution:}  \\
        1. Non-local similarity and global spectral low-rank property \\
\subsubsection{Neural Rendering in the wild}
    1. {\bf Solution:} Train Neural Render. Multiple stage training \\
\subsubsection{GeoNet: Deep Geodesic Networks for Point Cloud Analysis}